% !TEX TS-program = pdflatex
% !TEX encoding = UTF-8 Unicode

% This is a simple template for a LaTeX document using the "article" class.
% See "book", "report", "letter" for other types of document.

\documentclass[11pt]{article} % use larger type; default would be 10pt

\usepackage[utf8]{inputenc} % set input encoding (not needed with XeLaTeX)

%%% Examples of Article customizations
	% These packages are optional, depending whether you want the features they provide.
	% See the LaTeX Companion or other references for full information.

%%% PAGE DIMENSIONS
\usepackage{geometry} % to change the page dimensions
	\geometry{letterpaper} % or letterpaper (US) or a5paper or....
	\geometry{margin=1in} % for example, change the margins to 2 inches all round
	% \geometry{landscape} % set up the page for landscape
	%   read geometry.pdf for detailed page layout information

\usepackage{graphicx} % support the \includegraphics command and options
\usepackage{sidecap}

% \usepackage[parfill]{parskip} % Activate to begin paragraphs with an empty line rather than an indent

%%% PACKAGES
\usepackage{booktabs} % for much better looking tables
\usepackage{array} % for better arrays (eg matrices) in maths
\usepackage{paralist} % very flexible & customisable lists (eg. enumerate/itemize, etc.)
\usepackage{verbatim} % adds environment for commenting out blocks of text & for better verbatim
\usepackage{subfig} % make it possible to include more than one captioned figure/table in a single float
%\usepackage{datetime} %suggested to get todays data and time
% These packages are all incorporated in the memoir class to one degree or another...

\usepackage{listings} % For displaying code

\usepackage{placeins}      % This package gives us \FloatBarrier

\usepackage{mathtools} % this will also load amsmath package

%%% HEADERS & FOOTERS
\usepackage{fancyhdr} % This should be set AFTER setting up the page geometry
\pagestyle{fancy} % options: empty , plain , fancy
\renewcommand{\headrulewidth}{0pt} % customise the layout...
\lhead{}\chead{DSCI 351-351M-451 Syllabus: Exploratory Data Science}\rhead{}
%\lfoot{}\cfoot{\thepage}\rfoot{}
\lfoot{\today}\cfoot{}\rfoot{\thepage}

%%% SECTION TITLE APPEARANCE
\usepackage{sectsty}
\allsectionsfont{\sffamily\mdseries\upshape} % (See the fntguide.pdf for font help)
% (This matches ConTeXt defaults)

%\usepackage[labelfont = footnotesize, textfont = small, justification=centering]{caption}  %allows us to change the font of the captions on figures
%\setlength{\abovecaptionskip}{5pt}
%\setlength{\belowcaptionskip}{-10pt}

\usepackage{wrapfig}  %allows figures to have text wrapped around them. Apparently this does not work perfectly in Latex and requires manual adjustment of figures and such

\hyphenation{op-tical net-works semi-conduc-tor}
%
%\usepackage{parskip}
%\setlength{\parindent}{5pt}

\usepackage{enumitem}  %%%%allows you to begin enumerate at 0

%%%Allows you to have \textsubscript as a command
\usepackage{fixltx2e}   

%%% ToC (table of contents) APPEARANCE
\usepackage[nottoc,notlof,notlot]{tocbibind} % Put the bibliography in the ToC
\usepackage[titles,subfigure]{tocloft} % Alter the style of the Table of Contents
\renewcommand{\cftsecfont}{\rmfamily\mdseries\upshape}
\renewcommand{\cftsecpagefont}{\rmfamily\mdseries\upshape} % No bold!

\newcommand{\squeezeup}{\vspace{-2.5mm}}

%%%must be the last package called
\usepackage{hyperref}
\hypersetup{colorlinks=true, linkcolor=blue, citecolor=blue, filecolor=blue, urlcolor=blue, pdftitle=DSCI351-351M-451, pdfauthor=RHF, pdfsubject=EMSE343-443, pdfkeywords=}

%%% END Article customizations

%%% The "real" document content comes below...

\title{Biology 364/664 Syllabus \\ \emph{Analysis and Visualization of Biological Data} \\ Spring 2019 	\\
Tues Thurs 11:00am to 11:52am	and Tues 1:00pm to 4:52pm}
\author{Prof. Ken Field}
%\date{} % Activate to display a given date or no date (if empty),
         % otherwise the current date is printed 

\begin{document}
\maketitle

%\begin{figure} [!htbp]
%\centering
%	\includegraphics[height=.4\textheight]{figs/1501DSCI351Syllabus}
%	\caption{Syllabus, concepts and organization} 
%	\label{fig:Syllabus}
%\end{figure}


%-------------------------------------------------------
\section{Contact Information}

  Prof. Ken Field
  \begin{itemize}
  	\item 208 Biology Building
  	\item kfield@bucknell.edu
  	\item @ProfKenField on Twitter
  	\item @KField-Bucknell on GitHub
  \end{itemize}
  
  
%-------------------------------------------------------
\section{Course Description}

Introduction to the process of data exploration and visualization using
state-of-the-art computational techniques. Using ``big data'' from
public archives or their own research projects, students will learn how
to rigorously analyze and visualize complex biological datasets. Lab
will include hands-on work with R and virtual reality. No programming
experience required.
    
    
%-------------------------------------------------------
\section{Course Objectives}\label{course-objectives}}

\begin{enumerate}
\def\labelenumi{\arabic{enumi}.}
\item
  Students will analyze, visualize, and intepret real-world
  datasets using reproducible data science methods and R, R markdown, and Git.
\item
  Students will learn to identify and avoid questionable research practices 
  when designing experiments, analyzing data, and presenting results.
\item
  Working as a team, students will complete novel projects utilizing
  whole-transcriptome or whole-genome datasets.
\item
  Students will present their final projects using complex
  multi-dimensional data visualizations.
\end{enumerate}

\section{Grading}

  {\bf BIOL 364/664 is graded on 500 points basis}
  \begin{align*}
    Ten \, Homeworks, \, worth \, 10 \, points \, each = 100 pts.\\
    Four \, Data \, Projects, \, worth \, 25 \, points \, each = 100 pts.\\
    Takehome \, Midterm \, Exam = 100 pts.\\
    Final \, Project = 100 pts.
    Takehome \, Final \, Exam = 100 pts. \\
    {\bf Total = 500 pts.}
  \end{align*}

  The Data Projects and Final Project will be graded using 

\section{Textbooks and Readings}

  Required Texts and their Abbreviation, which is used in the syllabus:
  
  Peng R Programming (PRP) and Peng Exploratory Data Analysis (EDA) are introductory books to R and Data Science and Analysis. 
  These are  Leanpub books, available from LeanPub for a ”pay what you want” price. 
  
  R for Data Science (R4DS) is a new book teaching R for Tidy Data Science, and is available as a bookdown book on the web \href{http://r4ds.had.co.nz/}{R for Data Science}, you can buy an ebook for \$18 from \href{https://play.google.com/store/books/details/Hadley_Wickham_R_for_Data_Science?id=I6y3DQAAQBAJ}{Google Play Store page for this book}. 
  
  Open Intro Statistics version 3 (OIS) is an open source text book on Inferential Statistics, published under a Creative Commons license, \href{https://drive.google.com/file/d/0B-DHaDEbiOGkc1RycUtIcUtIelE/view}{for free distribution as a pdf}. 
  In addition a copy can be purchased from Amazon for \$9. 
  
  Introduction to Statistical Learning with R (ISLR) is a \href{"http://www.springer.com/us/book/9781461471370"}{Springer} book which is also available for  \href{http://www-bcf.usc.edu/~gareth/ISL/ISLR\%20Seventh\%20Printing.pdf}{free as a pdf}. ISLR is the text book used extensively in DSCI353/453 on Statistical Learning. 
  
  %\FloatBarrier
  
  \begin{SCfigure}
  	\centering
  	\caption{{\bf PRP}: Roger Peng, {\bf \textsc{R Programming for Data Science}}. 2014 \cite{peng_r_2014}}
  	\includegraphics[width=0.2\textwidth]%
  	{figs/Peng-R-Programming.png}% picture filename
  \end{SCfigure}
  
  \begin{SCfigure}
  	\centering
  	\caption{{\bf EDA}: Roger Peng, {\bf Exploratory Data Analysis With R}. 2015 \cite{peng_exploratory_2015}}
  	\includegraphics[width=0.2\textwidth]%
  	{figs/Peng-EDAwithR.png} % picture filename
  \end{SCfigure}
  
  \begin{SCfigure}
  	\centering
  	\caption{{\bf OIS}: David M. Diez, Christopher D. Barr, and Mine Cetinkaya-Rundel, \href{"https://www.openintro.org/stat/textbook.php"}{ {\bf OpenIntro Statistics 3rd Ed.} } 2015 \cite{david_m._diez_openintro_2015} }
  	\includegraphics[width=0.2\textwidth]%
  	{figs/OISv3.jpg} % picture filename
  \end{SCfigure}
  
  \begin{SCfigure}
      \centering
      \caption{{\bf R4DS}: Garrett Grolemund, Hadley Wickham {\bf R for Data Science}. 2017 \cite{wickham_r_2017}}
      \includegraphics[width=0.2\textwidth]%
      {figs/R4DScover.png}% picture filename
  \end{SCfigure}
  
  \begin{SCfigure}
  	\centering
  	\caption{ {\bf ISLR}: Gareth James, Daniela Witten, Trevor Hastie, Robert Tibshirani {\bf An Introduction to Statistical Learning: with Applications in R}, 2013  \cite{james_introduction_2013} }
  	\includegraphics[width=0.2\textwidth]%
  	{figs/ISLR.jpg} % picture filename
  \end{SCfigure}
  
  Additional reading assignments will be distributed via the course git repository in the readings subdirectory. 

%-------------------------------------------------------
\FloatBarrier
\section{DSCI351-451 Syllabus: Weekly Topics}
\begin{table}[h] 
	\centering % used for centering table 
	\begin{tabular}{| c | p{3.4cm} | p{3.4cm} | c | c |} % centered columns (5 columns) 
	\hline %inserts horizontal line
	Day:Date & Foundation & Practicum  & Reading & Due  \\ % inserts table heading 
	\hline 
	\hline % inserts double horizontal line 
%	 w1\addtolength{}{•}
  w1a:Tu:8/28/18 & ODS Tool Chain & R, Rstudio, Git & & \\ % inserting body of the table 
	\hline %inserts single line 
	w1b:Th:8/30/18 & Setup ODS Tool Chain & Bash, Git, Twitter & PRP4-33 & HW1 \\ 
	\hline  
	\hline
	w2a:Tu:9/4/18 & What is Data Science & OIS:Intro2R  & PRP35-64 & {\bf HW1 Due} \\ 
	\hline
	w2b:Th:9/6/18 &  Data Analytic Style, Git & 451SempProj, Git & PRP65-93, OI1-1.9 & HW2 \\
	\hline 
	\hline
	w3a:Tu:9/11/18* & Struct. of Data Analysis & ISLR:Intro2R, Loops & PRP94-116, OIS3 & {\bf HW2 Due}\\ 
	\hline
	w3b:Th:9/13/18* & OIS3 Intro to Data & GapMinder, Dplyr, Magrittr &  &  \\
	\hline
	\hline 
	w4a:Tu:9/18/18 & OIS3, Intro2Data part 2, Data & EDA: PET Degr. & EDA1-31 & Proj1  \\ 
	\hline
	w4b:Th:9/20/18 & Hypothesis Testing & GGPlot2 Tutorial & EDA32-58 & HW3 \\
	\hline
	\hline 
	w5a:Tu:9/25/18 & Distributions & SemProj RepOut1  & R4DS1-3 &  {\bf HW3 Due} \\ 
	\hline
	w5b:Th:9/27/18 & Wickham DSCI in Tidyverse & SemProj RepOut1 & R4DS4-6  & {\bf SemProj1}, \\
	\hline
	\hline 
	w6a:Tu:10/2/18 & OIS Found. of Inference & Inference  & R4DS7-8 & {\bf Proj1 Due} \\ 
	\hline
	w6b:Th:10/4/18 &  & Midterm Review & R4DS9-16 Wrangle & \\
	\hline
	\hline 
	w7a:Tu:10/9/18* & Summ. Stats \& Vis.  & Data Wrangling &  & \\ 
	\hline
	w7b:Th:10/11/18* & {\bf MIDTERM EXAM} &  &  & HW4 \\
	\hline
	\hline 
	w8a:Tu:10/16/18 & Numerical Inference &  Tidy Check Explore  & OIS4 & {\bf HW4 Due}\\ 
	\hline
	w8b:Th:10/18/18 & Algorithms, Models & Pairwise Corr. Plots  & OIS5.1-4 & Proj 2, HW5 \\
	\hline
	\hline 
	Tu:10/23 & {\bf CWRU FALL BREAK} &  & R4DS17-21 Program \\ 
	\hline
	w9b:Th:10/25/18 & Categorical Infer  & Predictive Analytics & OIS6.1,2 & \\ 
	\hline 
	\hline
	w10a:Tu:10/30/18 & SemProj & SemProj & OIS7   &  {\bf SemProj2}  {\bf HW5 Due}  \\
	\hline
	w10b:Th:11/1/18 & Lin. Regr. & Lin. Regr. &  OIS8 &  {\bf Proj.2 due} \\	
	\hline
	\hline
	w11a:Tu:11/6/18 & Inf. for Regression  & Curse of Dim. & OIS8 & Proj 3  \\
	\hline 
	w11b:Th:11/8/18 & Model Accuracy & Training Testing & ISLR3 & HW6  \\ 
	\hline
	\hline
	w12a:Tu:11/13/18 & Multiple Regr. & Mul. Regr. \& Pred. & ISLR4 & {\bf HW6 due}  \\
	\hline 
	w12b:Th:11/15/18 & Classification &  & ISLR6  &   \\ 
	\hline
	\hline
	w13a:Tu:11/20/18 & Classification & Clustering & ISLR5 & {\bf } {\bf Proj 3 due} \\
	\hline 
	Th:11/22/18 & {\bf THANKSGIVING} &  &   &  Proj 4 \\ 
	\hline
	\hline 
	w14a:Tu:11/27/18 & Big Data & Hadoop & &  \\ 
	\hline
	w14b:Th:11/29/18 & InfoSec & VerisDB &  & {\bf SemProj3} \\
	\hline
	w15a:Tu:12/4/18 & SemProj ReportOut3  &  & {\bf } \\ 
	\hline
	w15b:Th:12/6/18 & SemProj ReportOut3 &  & & {\bf Proj4} \\
	\hline
	\hline
	\hline
		  & {\bf FINAL EXAM} & {\bf Monday12/17, 12:00-3:00pm}  & Olin 313 & {\bf SemProj4 due} \\
	\hline
	\hline
\end{tabular}
 
\caption{DSCI351-451 Weekly Syllabus. Peng R Programming (PRPx.y), Peng Exploratory Data Analysis (EDAx.y), R for Data Science (R4DSx.y), Open Intro Statistics (OISx.y) and Introduction to Statistical Learning with R (ISLRx.y) refers to chapters and sections assigned as reading. }
\label{table:Syllabus} % is used to refer this table in the text 
\end{table} 

\FloatBarrier

%-------------------------------------------------------
\section{Course Mechanics}

  \subsection{Lectures}
  
    Fall 2017 	Tuesday, Thursday 11:30 am to 12:45 pm		Olin 303
  
  \subsection{Consultations}
  
    After class or as needed. Contact Prof. French and Alan Curran, email or in person. 
  
  \subsection{Homework Assignments}
  
    All homeworks assignments are submitted electronically through blackboard, uploading to the HW assignment page. 
  
    Filenames should contain DSCI351, -HW\#, -YourLastName... e.g. DSCI351-HW2-French.R or DSCI351-HW2-French.R and DSCI351-HW2-French.pdf. 
  
    Homeworks need to be legible, organized and explain your thinking, process and results. 
    Credit all resources you drew upon, including texts, papers, peers. 
  
    Homeworks are due by 11 am Tuesday, prior to the beginning of class. 
    Homeworks will be graded on blackboard and reviewed in class. 

%-------------------------------------------------------

\section{Coding and Data Science Tools and Resources}

  {\bf Open Data Science (ODS) VDIs} \\
  
    You will not need to install software on your personal computers. 
    
    Instead you can install the Citrix Reciever \cite{citrix_citrix_2014} and then login to the CWRU CSE Portal. \cite{cse_portal_cwru_2014} 
    
    The CSE Portal is located at \href{"https://cseportal.cwru.edu/vpn/index.html"}{https://cseportal.cwru.edu/vpn/index.html}    \\
    
    {\bf Scripting, Coding and Writing} \\
    And more resources for open science coding and scripting, including tools for code editing, code version control and languages. 
    
    {\bf R Statistics} \\
    We will be using R in this class for homeworks and projects. 
    Its generally useful language for statistical analysis and data science. 
  
      \begin{itemize}
        \item \href{"http://www.r-project.org/index.html"}{The R Project for Statistical Computing}  \cite{r_r_2014} main website
        \item \href{"http://en.wikipedia.org/wiki/R_(programming_language)"}{R programming language}  R is a free software programming language and software environment for statistical computing and graphics.\cite{r_project_r_2014} 
        \item \href{"http://www.rstudio.com/"}{ RStudio} provides popular open source and enterprise-ready professional software for the R statistical computing environment. \cite{rstudio_rstudio_2014}
        \item \href{"https://google-styleguide.googlecode.com/svn/trunk/Rguide.xml"}{ Google's R Style Guide}
      \end{itemize}
    
    {\bf Rmarkdown} as a path to open access and reproducible science
    
      \begin{itemize}
        \item \href{"http://rmarkdown.rstudio.com/"}{ R Markdown — Dynamic Documents for R.} We will be doing all our work using Rmarkdown this semester. 
        Class presentations, homeworks, projects, all done in Rmd, as reproducible science projects, including data, code, and final output. 
        \item \href{"http://shiny.rstudio.com/articles/rmarkdown.html"}{ Introduction to R Markdown.}
        \item \href{"http://www.rstudio.com/resources/cheatsheets/"}{ R Markdown Cheat Sheets.}
        \item \href{"http://rpubs.com/mansun_kuo/24330"}{ An Rmarkdown Introduction slidedeck done from Rmarkdown and shared publicly on RPubs.}
      \end{itemize}
    
    {\bf R Statistics, more resources} \\
              We will be using R in this class for homeworks and projects. Its generally useful language for statistical analysis and data science. 
      \begin{itemize}
        \item \href{"http://www.r-project.org/index.html"}{The R Project for Statistical Computing}  \cite{r_r_2014} main website
        \item \href{"https://www.youtube.com/user/rdpeng/playlists"}{Roger Peng's Computing for Data Analysis introduction to R Statistics}. 
        These are from a \href{"https://www.coursera.org/course/compdata"} {Coursera course} he does, with the same name. \cite{peng_computing_2014}\item \href{"http://cran.r-project.org/doc/contrib/Torfs+Brauer-Short-R-Intro.pdf"}{A (very) short introduction to R}  \cite{torfs_very_2014}
        \item \href{"https://google-styleguide.googlecode.com/svn/trunk/Rguide.xml"}{ Google's R Style Guide}
        \item \href{"http://stat405.had.co.nz/r-style.html"}{ Hadley Wickham's R Style Guide}  
        \item \href{"http://www.rstudio.com/resources/cheatsheets/"}{ RStudio's R Cheatsheets for Rmarkdown and Data Wrangling}
        \item \href{"http://www.theresearchkitchen.com/blog"}{An Rmd slideshow Intro to R}
      \end{itemize}
    
    {\bf Open Source software and tools} \\
    
      \begin{itemize}
        \item \href{"http://en.wikipedia.org/wiki/Portal:Free_software"}{FOSS (Free and Open Source Software)} is a copyleft approach to software which is hat is distributed in a manner that allows its users to run the software for any purpose, to redistribute copies of, and to examine, study, and modify, the source code. \cite{_portal:free_2014}
        \item \href{"http://www.vim.org/"}{vim (or Gvim the gui version)} is a powerful text and code editor, that is universally available on all linux and mac computers.\cite{_gvim_2014}   \href{"http://neovim.org/"}{NeoVim} is a new Gvim fork.\cite{_neovim_2014} It can be installed on windows computers, its available on the ODS VDIs.. \cite{_gvim_2014}
        \item \href{"http://en.wikipedia.org/wiki/Git_(software)"}{Git (Wikipedia)} is a distributed content versioning system that is very popular. It enables collaborative code development and LaTeX writing projects.\cite{_git_2014-2} 
        \item \href{"http://git-scm.com/"}{Git server software} is installed on each computer.\cite{_git_2014}  
        \item  \href{"https://github.com/"}{GitHub} is a Git server website used for collaborative code development.\cite{_github_2014}
        \item  \href{"https://bitbucket.org/"}{BitBucket} is a Git server website used for collaborative code development. If you join with your case.edu email address, you get unlimited private repositories.\cite{_bitbucket:_2014} 
        \item \href{"http://stackexchange.com/tour"}{Stack Exchange}  \cite{stack_exchange_stack_2014} Code Question and Answer Websites: covering R, Python, Mathematica, {LaTeX} and many other things, such as English or Spanish etc.
      \end{itemize}
      
    {\bf Python (is also used for Data Science in many cases. But here we will focus on R first. } 

      \begin{itemize}
        \item \href{"https://en.wikipedia.org/wiki/Python_(programming_language)"}{Wikipedia: Python is a widely used general-purpose, high-level programming language.}  \cite{python_python_2014}
        \item \href{"https://www.python.org/"}{The Python main website.} \cite{python_python.org_2013}
        \item \href{"https://docs.python.org/2/tutorial/"}{The Python Tutorial — Python v2.7.8 documentation}  \cite{python_python_2014}
        \item \href{"http://docs.python-guide.org/en/latest/"}{The Hitchhikers Guide to Python}. This is an open access book being hosted on developed on \href{"https://github.com/"}{GitHub} and is located here \href{"https://github.com/vuvlab/python-guide"}{https://github.com/vuvlab/python-guide}. \cite{_hitchhikers_2014} \cite{_kennethreitz/python-guide_2014}
        \item \href{"http://www.numpy.org"}{NumPy is the fundamental package} \cite{numpy_numpy_2014} for scientific computing with Python.
        \item \href{"http://www.ctcms.nist.gov/fipy/"}{FiPy: Partial Differential Equations with Python} \cite{guyer_fipy:_2009}
        \item \href{"http://www.scipy.org/"}{SciPy is a python-based ecosystem}  \cite{scipy_scipy.org_2014} of open-source software for mathematics, science, and engineering. 
        \item \href{"https://code.google.com/p/pythonxy/"}{PythonXY - Scientific-oriented Python Distribution} based on Qt and Spyder that runs on Windows. \cite{pythonxy_pythonxy_2014}
        \item \href{"http://ipython.org/index.html"}{IPython Shell and Notebook}  \cite{ipython_ipython_2014}
        \item \href{"https://code.google.com/p/spyderlib/"}{Spyder is the Scientific PYthon Development Environment}  \cite{spyder_spyder_2014}
      \end{itemize}
    
    {\bf LaTeX is used for publication quality writing. Its also the backend for Rmarkdown's pdf generation. It lets you write professional looking papers, theses and books, along with presentations. } 
      \begin{itemize}
        \item \href{"http://www.tug.org/"}{LaTeX} is a program for writing documents, paper, journal articles, presentations and theses. \cite{_tex_2014}
        \item \href{"http://en.wikibooks.org/wiki/LaTeX"}{LaTeX - Wikibooks}, open books for an open world. \cite{latex_latex_2014}
        \item \href{"https://www.zotero.org/"}{Zotero Reference-Citation Manager, BibTeX Client}  \cite{zotero_zotero_2014}
      \end{itemize}

%-------------------------------------------------------
\section{Policies}

  \subsection{Attendance}
  
    You attendance is expected. 
    Some information is covered that is not in the text. 
    Student participation is an important part of the class. 
  
  \subsection{Readings}
  
    Readings must be done, BEFORE the class, where they are assigned. 
    The reading assignment, is for the class with which it is listed. 
  
  \subsection{Homework Assignments}
  
    Homeworks are due before noon on Monday after the week they are assigned.
    Homework assignments will be submitted on GitHub after week 1.
    A 50\% deduction will be assessed for submissions not received by noon on Monday. 
  
  \subsection{Collaboration and Citation}
  
    For all projects and homework assignments working together is acceptable and encouraged. 
    Working together and discussion is not allowed on takehome exams.
    It is not ethical to do someone else's work or to have someone do your work. 
    You must cite \textbf{all} resources used to work on your homework and projects. 
    Citations should be done at the end of the document. 
    These references can be to books, Wikipedia and other web resources, and discussions with other students. 
  
  \subsection{Academic Integrity Policy}
  
    All students in this course are expected to adhere to University standards of academic integrity. Cheating, plagiarism, misrepresentation, and other forms of academic dishonesty will not be tolerated. 
    This includes, but is not limited to, consulting with another person during an exam, turning in written work that was prepared by someone other than you, making minor modifications to the work of someone else and turning it in as your own, or engaging in misrepresentation in seeking a postponement or extension. 
    Ignorance will not be accepted as an excuse. 
    If you are not sure whether something you plan to submit would be considered either cheating or plagiarism, it is your responsibility to ask for clarification.  
    
    For complete information, please go to \href{"https://students.case.edu/community/conduct/aiboard/policy.html"}{https://students.case.edu/community/conduct/aiboard/policy.html}. 
  
  \subsection{Disability Resources}
  
    ESS Disability Resources is committed to assisting all CWRU students with disabilities by creating opportunities to take full advantage of the University's educational, academic, and residential programs.  
    
    For further information, please go to \href{"https://students.case.edu/academic/disability/''}{https://students.case.edu/academic/disability/}. 


%-------------------------------------------------------
\section{Copyleft, References, Citations  \& Rubrics}

  \subsection{CopyLeft}
  
    Creative Commons plays an important role in openness and open science, open data, open source efforts. 
    
    This class is covered by a \href{"http://creativecommons.org/licenses/"}{Creative Commons} license. 
    
    The license we'll use for class materials, code and presentations is covered by  the "Attribution-ShareAlike 4.0 International" license, which is commonly called the CC BY-SA 4.0 license. \cite{_creative_2015}

%-----------------------------------------

\bibliographystyle{ieeetr}
\bibliography{dsci351-451}

\end{document}
