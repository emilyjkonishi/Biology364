% !TEX TS-program = pdflatex
% !TEX encoding = UTF-8 Unicode

% This is a simple template for a LaTeX document using the "article" class.
% See "book", "report", "letter" for other types of document.

\documentclass[11pt]{article} % use larger type; default would be 10pt

\usepackage[utf8]{inputenc} % set input encoding (not needed with XeLaTeX)

%%% Examples of Article customizations
	% These packages are optional, depending whether you want the features they provide.
	% See the LaTeX Companion or other references for full information.

%%% PAGE DIMENSIONS
\usepackage{geometry} % to change the page dimensions
	\geometry{letterpaper} % or letterpaper (US) or a5paper or....
	\geometry{margin=0.75in} % for example, change the margins to 2 inches all round
	% \geometry{landscape} % set up the page for landscape
	%   read geometry.pdf for detailed page layout information

\usepackage{graphicx} % support the \includegraphics command and options
\usepackage{sidecap}

% \usepackage[parfill]{parskip} % Activate to begin paragraphs with an empty line rather than an indent

%%% PACKAGES
\usepackage{booktabs} % for much better looking tables
\usepackage{array} % for better arrays (eg matrices) in maths
\usepackage{paralist} % very flexible & customisable lists (eg. enumerate/itemize, etc.)
\usepackage{verbatim} % adds environment for commenting out blocks of text & for better verbatim
\usepackage{subfig} % make it possible to include more than one captioned figure/table in a single float
%\usepackage{datetime} %suggested to get todays data and time
% These packages are all incorporated in the memoir class to one degree or another...

\usepackage{listings} % For displaying code

\usepackage{placeins}      % This package gives us \FloatBarrier

\usepackage{mathtools} % this will also load amsmath package

%%% HEADERS & FOOTERS
\usepackage{fancyhdr} % This should be set AFTER setting up the page geometry
\pagestyle{fancy} % options: empty , plain , fancy
\renewcommand{\headrulewidth}{0pt} % customise the layout...
\lhead{}\chead{BIOL 364/664 Syllabus}\rhead{}
%\lfoot{}\cfoot{\thepage}\rfoot{}
\lfoot{\today}\cfoot{}\rfoot{\thepage}

%%% SECTION TITLE APPEARANCE
\usepackage{sectsty}
\allsectionsfont{\sffamily\mdseries\upshape} % (See the fntguide.pdf for font help)
% (This matches ConTeXt defaults)

%\usepackage[labelfont = footnotesize, textfont = small, justification=centering]{caption}  %allows us to change the font of the captions on figures
%\setlength{\abovecaptionskip}{5pt}
%\setlength{\belowcaptionskip}{-10pt}

\usepackage{wrapfig}  %allows figures to have text wrapped around them. Apparently this does not work perfectly in Latex and requires manual adjustment of figures and such

\hyphenation{op-tical net-works semi-conduc-tor}
%
%\usepackage{parskip}
%\setlength{\parindent}{5pt}

\usepackage{enumitem}  %%%%allows you to begin enumerate at 0

%%% ToC (table of contents) APPEARANCE
\usepackage[nottoc,notlof,notlot]{tocbibind} % Put the bibliography in the ToC
\usepackage[titles,subfigure]{tocloft} % Alter the style of the Table of Contents
\renewcommand{\cftsecfont}{\rmfamily\mdseries\upshape}
\renewcommand{\cftsecpagefont}{\rmfamily\mdseries\upshape} % No bold!

\newcommand{\squeezeup}{\vspace{-2.5mm}}

%%%must be the last package called
\usepackage{hyperref}
\hypersetup{colorlinks=true, linkcolor=blue, citecolor=blue, filecolor=blue, urlcolor=blue, pdftitle=DSCI351-351M-451, pdfauthor=RHF, pdfsubject=EMSE343-443, pdfkeywords=}

%%% END Article customizations

%%% The "real" document content comes below...

\title{Biology 364/664 Syllabus \\ \emph{Analysis and Visualization of Biological Data} \\ Spring 2019 	\\
Tues Thurs 11:00am to 11:52am	and Tues 1:00pm to 4:52pm}
\author{Prof. Ken Field}
%\date{} % Activate to display a given date or no date (if empty),
         % otherwise the current date is printed 

\begin{document}
\maketitle

%\begin{figure} [!htbp]
%\centering
%	\includegraphics[height=.4\textheight]{figs/1501DSCI351Syllabus}
%	\caption{Syllabus, concepts and organization} 
%	\label{fig:Syllabus}
%\end{figure}


%-------------------------------------------------------
\section{Contact Information}

  Prof. Ken Field
  \begin{itemize}
  	\item 208 Biology Building
  	\item kfield@bucknell.edu
  	\item @ProfKenField on Twitter
  	\item @KField-Bucknell on GitHub
  \end{itemize}
  
  
%-------------------------------------------------------
\section{Course Description}

Introduction to the process of data exploration and visualization using
state-of-the-art computational techniques. Using ``big data'' from
public archives or their own research projects, students will learn how
to rigorously analyze and visualize complex biological datasets. Lab
will include hands-on work with R and virtual reality. No programming
experience required.
    
    
%-------------------------------------------------------
\section{Course Objectives}\label{course-objectives}

\begin{enumerate}
\def\labelenumi{\arabic{enumi}.}
\item
  Students will analyze, visualize, and intepret real-world
  datasets using reproducible data science methods and R, R markdown, and Git.
\item
  Students will learn to identify and avoid questionable research practices 
  when designing experiments, analyzing data, and presenting results.
\item
  Working as a team, students will complete novel projects utilizing
  whole-transcriptome or whole-genome datasets.
\item
  Students will present their final projects using complex
  multi-dimensional data visualizations.
\end{enumerate}

\section{Grading}

  \begin{align*}
    Eight \, Homeworks, \, worth \, 12.5 \, points \, each = 100 pts.\\
    Four \, Data \, Projects, \, worth \, 25 \, points \, each = 100 pts.\\
    Takehome \, Midterm \, Exam = 100 pts.\\
    Final \, Project = 200 pts.\\
    Takehome \, Final \, Exam = 100 pts.\\
    {\bf Total = 600 pts.}
  \end{align*}

  The Data Projects and Final Project will be graded using labor-based contract grading as described by \href{http://www.translingualwriting.com/resources/Inoue\%20Contract.pdf}{Asao Inoue}. These assignments will utilize a goal-oriented grading system as you develop your skills as a data scientist.  When you or your group completes all of the goals associated with each project, you will earn 21/25 or 170/200 pts (85\%). Failure to do so will result in a lower grade.

To earn more than 85\%, you must do the following:
\begin{itemize}
  \item Propose and perform additional analysis and visualization comparisons (discuss available options with me).
  \item Routinely assist in the learning and proficiency of your peers.
\end{itemize}


\section{Textbooks and Readings}

  Required Texts and their Abbreviations, which are used on the syllabus:
  \begin{itemize}
 
    \item A Primer in Biological Data Analysis and Visualization Using R, by Gregg Hartvigsen, ISBN: 9780231166997 (Hart)
  
    \item \href{https://leanpub.com/rprogramming}{R Programming for Data Science}, by Roger D. Peng (PRP)
  
    \item \href{https://leanpub.com/exdata}{Exploratory Data Analysis}, by Roger D. Peng (EDA) 
  
  These are  Leanpub books, available from LeanPub for a ”pay what you want” price. 
  
    \item R for Data Science (R4DS) is a book teaching R for Tidy Data Science, and is available as a bookdown book on the web \href{http://r4ds.had.co.nz/}{R for Data Science}, you can buy an ebook for \$18 from \href{https://play.google.com/store/books/details/Hadley_Wickham_R_for_Data_Science?id=I6y3DQAAQBAJ}{Google Play Store page for this book}. 
  
    \item Open Intro Statistics version 3 (OIS) is an open source text book on Inferential Statistics, published under a Creative Commons license, \href{https://drive.google.com/file/d/0B-DHaDEbiOGkc1RycUtIcUtIelE/view}{for free distribution as a pdf}. In addition a copy can be purchased from Amazon for \$9. 
  
    \item Additional reading assignments will be distributed via the course git repository in the Readings subdirectory. 
  \end{itemize}

%-------------------------------------------------------
\FloatBarrier
\section{BIOL 364 Syllabus: Weekly Topics}
\begin{table}[ht] 
	\centering % used for centering table 
	\begin{tabular}{| c | p{5cm} | p{3cm} | p{5.5cm} |} % centered columns (5 columns) 
	\hline %inserts horizontal line
	Week:Date & Topic & Reading & Due  \\ % inserts table heading 
	\hline 
	\hline % inserts double horizontal line 
%	 w1\addtolength{}{•}
  w1:15Jan2019 & Using R, Rstudio, Git & HART1,2,3, PRP4,5 & Making Graphs in R \\ % inserting body of the table 
	\hline
	w2:22Jan2019 & Simple Data Exploration & HART4 PRP6,7,8,9,10 EDA4,5,6,7 OIS1,2,3 & Data Exploration in R \\ 
	\hline
	w3:29Jan2019 & Data Visualization & HART5,11 EDA15,16 OIS4,5,6 & {\bf R Tutorial for 205 Project}\\ 
	\hline 
	w4:5Feb2019 & Advanded Hypothesis Testing & HART6,7,8 OIS7,8 & Experimental Design  \\ 
	\hline 
	w5:12Feb2019 & Questionable Data Practices & Fraser.pdf \newline Frostmeier.pdf & Multiple Testing \\ 
	\hline 
	w6:19Feb2019 & Questionable Data Practices  & Frostmeier.pdf & {\bf QRP Case Studies Project} \\ 
	\hline 
	w7:26Feb2019 & Transcriptomics &  & QC \& Mapping Statistics\\ 
	\hline 
	w8:5Mar2019 & Transcriptomics  &  & {\bf Transcriptomics Pipeline Project}\\ 
	\hline 
	12Mar2019 & {\bf SPRING BREAK} &  &  \\ 
	\hline
	w9:19Mar2019 & Interactive Data & EDA5 \newline minitufte.pdf & Data Archives \newline {\bf TAKEHOME MIDTERM} \\ 
	\hline
	w10:26Mar2019 & Final Project & &  {\bf Data Visualization Project} \\
	\hline
	w11:2Apr2019 & Final Project & EDA12,13,14 & Data Wrangling  \\
	\hline
	w12:9Apr2019 & Final Project &  & Preregistration  \\
	\hline
	w13:16Apr2019 & Final Project &  & {\bf Final Project Presentations} \\
	\hline 
	w14:23Apr2019 & Final Project & & {\bf Final Project Presentations}  \\ 
	\hline
	\hline
	\hline
	{\bf TBA}	  & {\bf TAKEHOME FINAL} &   &  \\
	\hline
	\hline
\end{tabular}
 
\caption{BIOL356/664 Weekly Syllabus. Peng R Programming for Data Science (PRPx), Peng Exploratory Data Analysis (EDAx), Open Intro Statistics (OISx) and A Primer in Biological Data Analysis and Visualization Using R (HARTx) refers to chapters assigned as reading. }
\label{table:Syllabus} % is used to refer this table in the text 
\end{table} 

\FloatBarrier

%-------------------------------------------------------


\section{Policies}

  \subsection{Attendance}
  
    Your attendance at all classes and lab is expected, but not a graded part of the course. If you will need to miss lab for any reason, contact Prof. Field before class to make arrangements.
  
  \subsection{Readings}
  
    Readings must be done BEFORE the class where they are assigned. For dates with multiple reading assignments, browse each to determine the sections that are most useful to you. There will be overlap between the various textbooks and you should choose the book that is best for you and your individual background.
  
  \subsection{Homework Assignments}
  
    Homeworks are due before noon on Friday on the week they are assigned.
    Homework assignments will be submitted on GitHub (more instructions to follow).
    A 25\% deduction will be assessed for submissions not received by noon on Friday. Assignments will not be accepated after noon on Monday.
  
  \subsection{Collaboration and Citation}
  
    For all projects and homework assignments working together is acceptable and encouraged. 
    It is not ethical to do someone else's work or to have someone do your work. 
    You must cite \textbf{all} resources used to work on your homework and projects. 
    Citations should be done at the end of the document. 
    These references can be to books, Wikipedia and other web resources, and discussions with other students. 
    Working together and discussion is not allowed on takehome exams.
  
  \subsection{Academic Integrity Policy}
  
    Read \href{"http://www.bucknell.edu/AcademicResponsibility.xml"}{Academic Responsibility at Bucknell} for policies regarding academic integrity. Any questions concerning academic responsibility or misconduct will be referred to the Board of Review for Academic Responsibility without hesitation. Always cite the source of any information from outside sources, including online sources and classmates. Assignments may be screened using software designed to detect plaigarism. Unless explicitly directed otherwise, all takehome exams are expected to represent individual, not collaborative, work.
  
  \subsection{Bucknell University Honor Code}
  \begin{enumerate}
  \item I will not lie, cheat, or steal in my academic endeavors.
  \item I will forthrightly oppose each and every instance of academic dishonesty.
  \item I will let my conscience guide my decision to communicate directly with any person or persons I believe to have been dishonest in academic work.
  \item I will let my conscience guide my decision on reporting breaches of academic integrity to the appropriate faculty or deans.
\end{enumerate}

  \subsection{Accommodations}
  
  Any student who needs an accommodation based on the impact of a disability should contact Heather Fowler, Director of the \href{https://www.bucknell.edu/about-bucknell/accessibility-resources/services-and-resources-for-students/academic-accommodations}{Office of Accessibility Resources} at hf007@bucknell.edu, 570-577-1188 or in room 212 Carnegie Building who will coordinate reasonable accommodations for students with documented disabilities. The college will make reasonable accommodations for persons with disabilities. 

%-------------------------------------------------------

\section{License}
  
    Creative Commons plays an important role in open science, open data, open source efforts. This class is covered by a \href{"http://creativecommons.org/licenses/"}{Creative Commons} license. The license we'll use for class materials, code and presentations is covered by  the "Attribution-ShareAlike 4.0 International" license, which is commonly called the CC BY-SA 4.0 license. Some of the materials for this course, including portions of the Syllabus, are derived from work by Roger H. French \href{"https://twitter.com/frenchrh"}{@frenchrh} Kyocera Professor, Materials Science, Case Western Reserve University.

%-----------------------------------------


\end{document}
